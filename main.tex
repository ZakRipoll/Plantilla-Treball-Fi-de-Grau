\documentclass[a4paper,12pt,twoside]{book}

%\usepackage{fontspec} Packages
\usepackage[catalan]{babel} % Idioma del document en català.
\usepackage[utf8]{inputenc}
\usepackage{emptypage} % Per tal que no surtin els números en les pàgines en blanc.
\usepackage{blindtext}
\usepackage[tmargin=3cm, lmargin=3cm, rmargin=3cm, bmargin=3cm, marginparsep=0.3cm, marginparwidth=2.4cm]{geometry}
\usepackage[sorting=none]{biblatex} % Per la bibliografia a més que els números surtin per l'ordre en què apareixen, no en l'ordre que s'ha citat al llarg del document.

\usepackage{import}

%\setmainfont{Arial}
\usepackage{graphicx}

\usepackage{subfiles} %Per importar la pàgina del títol

\usepackage{fancyhdr} %Que quedi maco

\pagestyle{fancy}
\fancyhf{}
\fancyhead[LE,RO]{\leftmark}
\fancyhead[RE,LO]{\thepage}

\pagenumbering{gobble}

%DEFINICIONS per tal de no haver d'escriure-ho cada vegada.

\input{Acronyms.tex}

%BIBLIOGRAFIA

\addbibresource{bibliografia.bib}


\begin{document}

%TITOL
\subfile{Titol}

\cleardoublepage % Neteja de pàgines, en ser un llibre ja la deixa en blanc en cas de necessitat.

\pagenumbering{Roman} % Nombre de pàgina en números Romans.

\chapter*{Agraïments} %* Per tal que no surti en el Table of contents, així la primera secció començarà amb la Introducció.
\addcontentsline{toc}{chapter}{Agraïments} % Per tal d'afegir a la Table of contents, amb el nom Agraïments.
\input{Chapters/Agraiments} % Carpeta i arxiu el qual es vol afegir, així el main només queda amb l'esquelet i el contingut va repartir en els diferents arxius.

\tableofcontents % Taula de continguts


\cleardoublepage

\listoffigures
\addcontentsline{toc}{chapter}{Índex de figures}

\cleardoublepage

\listoftables
\addcontentsline{toc}{chapter}{Índex de taules}

\cleardoublepage

\printacronyms[include-classes=abbrev,name=Abbreviations,heading=chapter*]
\addcontentsline{toc}{chapter}{Abbreviations}

\cleardoublepage

\pagenumbering{arabic} % Nombre de la pàgina en números Àrabs.

\chapter{Introducció} % Ja es posen en la Taula de continguts.
\section{Seccions principals de l'article, vindria a ser el Title 2 del Google Docs.}
El Title 1 en aquest cas correspondri a l'article,  en el main principal.
\subsection{Subseccions com el Títol 3 del Google Docs.}

\newpage %Veure el fancy com queda

\subsubsection{Subseccions com el Títol 4 del Google Docs, no es pot tirar més avall en els TGFs de \acl{upf}...}
Totes aquestes seccions quedaran posades en la Taula de continguts \cite{autor} % Per tal de poder citar una des de la bibliografia.

\addcontentsline{toc}{chapter}{Bibliografia}

\printbibliography

\end{document}
